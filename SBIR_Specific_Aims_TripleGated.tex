\documentclass[11pt,letterpaper]{article}

% Packages
\usepackage[utf8]{inputenc}
\usepackage[margin=0.75in]{geometry}
\usepackage{times}
\usepackage{setspace}
\usepackage{graphicx}
\usepackage{hyperref}
\usepackage{enumitem}
\usepackage{bold-extra}
\usepackage[parfill]{parskip}

% Formatting
\setlength{\parindent}{0pt}
\setstretch{1.0}

\begin{document}

\pagestyle{empty}

% Title and Header
{\centering
\textbf{\Large Triple-Gated Plant Extracellular Vesicle--PLGA Nanoparticle Platform for Safer Cyclosporine A Therapy in Inflammatory Bowel Disease}\\[0.3cm]
}

\noindent\textbf{Principal Investigator:} Maria Beatriz Herrera Sanchez, PhD\\
\textbf{Institution:} ExoVitae Lab / [University/Company Name]\\
\textbf{Grant Mechanism:} NIH SBIR Phase I (NIDDK)\\
\textbf{Total Budget:} \$500,000 (12 months)

\vspace{0.3cm}
\hrule
\vspace{0.3cm}

% Background and Significance
\noindent\textbf{Overview and Significance.} Cyclosporine A (CsA) is an effective rescue therapy for steroid-refractory ulcerative colitis, achieving 60--80\% response rates in acute severe cases. However, clinical use is constrained by dose-limiting nephrotoxicity (20--40\% of patients), neurotoxicity, narrow therapeutic window, and high inter-individual pharmacokinetic variability. Current oral CsA formulations (microemulsions) lack gut-specific targeting, resulting in systemic exposure that drives toxicity while failing to maximize local mucosal immunosuppression.

Existing nanoformulation strategies address individual limitations: PLGA nanoparticles improve CsA solubility and enable sustained release; plant-derived extracellular vesicles (PDEVs) provide biocompatible oral carriers with natural gut tropism; and PLA2-cleavable phospholipid linkers enable inflammation-triggered drug release. However, \textbf{no prior work has integrated these three control mechanisms into a single, multi-compartment platform} that synergistically addresses targeting, activation, and kinetics.

\textbf{Central Hypothesis.} A \textbf{triple-gated delivery system}---CsA-loaded PLGA nanoparticles encapsulated within PDEVs and tethered via phospholipase A2 (PLA2)-cleavable lipid linkers---will enable: (i) preferential accumulation in inflamed intestinal tissue via PDEV tropism, (ii) enzyme-triggered nanoparticle release at disease sites where secreted PLA2 is elevated, and (iii) sustained local CsA delivery via PLGA biodegradation, resulting in \textbf{superior therapeutic index} (efficacy with $<$50\% systemic exposure) compared to conventional formulations.

\textbf{Objective.} To establish the feasibility, mechanism of action, and therapeutic potential of the \textbf{CsA-PLGA-EV-PLA2 platform} as a next-generation rescue therapy for inflammatory bowel disease (IBD).

% Specific Aims
\vspace{0.2cm}
\noindent\rule{\textwidth}{0.5pt}
\vspace{0.1cm}

\section*{SPECIFIC AIMS}

\subsection*{AIM 1: Develop and characterize CsA-loaded PLGA nanoparticles and their incorporation into plant-derived extracellular vesicles}

We will formulate CsA-loaded PLGA nanoparticles (NPs, 100--200 nm diameter) using nanoprecipitation or double-emulsion methods, optimizing for: (a) high drug loading (target: $\geq$5--10 wt\%), (b) controlled release kinetics (24--72 h), and (c) colloidal stability. In parallel, we will isolate PDEVs from citrus juice (\textit{Citrus sinensis}, \textit{Citrus paradisi}) and characterize particle size, zeta potential, protein/lipid composition, and EV markers (CD63, TSG101). We will then develop protocols for efficient encapsulation of CsA-PLGA NPs into PDEVs via co-incubation, mild sonication, or extrusion, quantifying:
\begin{itemize}[leftmargin=*, topsep=0pt, itemsep=0pt]
    \item Nanoparticle encapsulation efficiency (\# NPs per EV; \% of total NPs captured)
    \item Structural integrity by transmission electron microscopy (TEM)
    \item Stability in simulated gastric fluid (SGF, pH 1.2, 2h) and simulated intestinal fluid (SIF, pH 6.8, 6h)
    \item PLGA-EV hybrid size, polydispersity index (PDI), and zeta potential
\end{itemize}

\textbf{Expected Outcome:} Reproducible PDEV-PLGA-CsA formulations with $\geq$50\% NP encapsulation efficiency, preserved EV integrity (confirmed by TEM and marker retention), and $\geq$80\% stability in GI-relevant conditions. \textbf{Go/No-Go (Month 4):} Achieve all metrics across 3 independent batches.

\subsection*{AIM 2: Engineer PLA2-cleavable surface modifications and demonstrate enzyme-triggered nanoparticle release}

We will design phospholipid linkers with PLA2-sensitive sn-2 ester bonds that: (i) insert into PDEV membranes, and (ii) mediate association between PDEVs and internal PLGA NPs (e.g., via hydrophobic moiety anchoring to NP surface or via cross-linking chemistry). Candidate linkers will be based on validated PLA2-responsive lipids used in colon-targeted prodrug systems (e.g., palmitoyl-homoserine phospholipids). We will incorporate linkers into PDEVs post-isolation and quantify:

\textbf{Sub-aim 2a (Enzyme-triggered release):}
\begin{itemize}[leftmargin=*, topsep=0pt, itemsep=0pt]
    \item NP release kinetics in presence/absence of secreted PLA2 (sPLA2-IIA, recombinant enzyme)
    \item NP release in homogenates from inflamed vs. healthy colon tissue (DSS-colitis mice)
    \item Target: $\geq$2--3-fold increase in NP release with PLA2 vs. non-cleavable control linkers
\end{itemize}

\textbf{Sub-aim 2b (Cellular uptake and immunosuppressive activity):}
Compare free CsA, CsA-PLGA NPs, non-cleavable PDEV-PLGA, and PLA2-cleavable PDEV-PLGA in:
\begin{itemize}[leftmargin=*, topsep=0pt, itemsep=0pt]
    \item Human intestinal epithelial cells (Caco-2): uptake kinetics (fluorescence microscopy), transepithelial electrical resistance (TEER), tight junction proteins (ZO-1, occludin)
    \item Human T cells (PBMCs): IL-2 secretion (ELISA), NFAT nuclear translocation, proliferation (CFSE)
    \item Hypothesis: PLA2-cleavable PDEV-PLGA achieves equivalent suppression at $\leq$50\% CsA dose vs. free CsA
\end{itemize}

\textbf{Expected Outcome:} PLA2 activity selectively triggers NP release ($\geq$2-fold increase); PLA2-cleavable formulations show superior cellular uptake and immunosuppression vs. non-cleavable controls at equal CsA dose. \textbf{Go/No-Go (Month 8):} Enzyme-triggered release confirmed; IC$_{50}$ for IL-2 suppression $\leq$50\% of free CsA.

\subsection*{AIM 3: Validate multi-level targeting and therapeutic index in a murine colitis model}

We will evaluate oral administration in DSS-induced colitis (C57BL/6 mice, n=10/group):

\textbf{Groups:} (1) Vehicle, (2) Free CsA (10 mg/kg), (3) CsA-PLGA NPs, (4) Non-cleavable PDEV-PLGA-CsA, (5) \textbf{PLA2-cleavable PDEV-PLGA-CsA} (target: 5 mg/kg CsA-equivalent).

\textbf{Efficacy endpoints:}
\begin{itemize}[leftmargin=*, topsep=0pt, itemsep=0pt]
    \item Daily disease activity index (DAI): weight loss, stool consistency, fecal blood
    \item Terminal (Day 8): colon length, histology score (blinded pathologist), tissue cytokines (IL-1$\beta$, TNF-$\alpha$, IL-6)
    \item Immune profiling: Flow cytometry of lamina propria (CD4$^+$ T-cell subsets, macrophage phenotypes)
\end{itemize}

\textbf{Pharmacokinetics and safety:}
\begin{itemize}[leftmargin=*, topsep=0pt, itemsep=0pt]
    \item Plasma CsA concentration (2h, 6h post-dose; LC-MS/MS); calculate AUC$_{0-24h}$
    \item Biodistribution: Fluorescently labeled NPs/EVs; tissue CsA levels (colon, kidney, liver, spleen)
    \item Nephrotoxicity: Serum creatinine, blood urea nitrogen (BUN), kidney histology (tubular damage score)
    \item Hepatotoxicity: Alanine aminotransferase (ALT), aspartate aminotransferase (AST)
\end{itemize}

\textbf{Expected Outcome:} PLA2-cleavable PDEV-PLGA-CsA achieves: (i) $\geq$40\% reduction in DAI vs. vehicle (non-inferior to free CsA at 10 mg/kg), (ii) plasma CsA AUC $<$50\% of free CsA, (iii) $\geq$3-fold higher colonic CsA concentration vs. free CsA, and (iv) no significant increase in serum creatinine or kidney histopathology vs. vehicle. \textbf{Go/No-Go (Month 12):} Meet efficacy + safety criteria; demonstrate PLA2-dependent biodistribution advantage.

% Impact and Innovation
\vspace{0.2cm}
\noindent\rule{\textwidth}{0.5pt}
\vspace{0.1cm}

\section*{IMPACT \& INNOVATION}

\textbf{Innovation:} This is the \textbf{first integration} of PLGA nanoparticles, plant-derived extracellular vesicles, and enzyme-cleavable linkers into a single, multi-gated platform. Key innovations:

\begin{enumerate}[leftmargin=*, topsep=0pt, itemsep=2pt]
    \item \textbf{Triple-control architecture:} PDEV tropism (Level 1) + PLA2 activation (Level 2) + PLGA kinetics (Level 3) provide orthogonal mechanisms for spatiotemporal control of CsA delivery.
    \item \textbf{Inflammation-responsive release:} PLA2-cleavable linkers exploit the elevated sPLA2 activity in IBD mucosa, creating a ``smart'' formulation that releases drug preferentially at disease sites.
    \item \textbf{Nested nanoparticle concept:} Protects PLGA NPs through harsh GI transit; enhances local accumulation; reduces systemic exposure more effectively than single-component systems.
    \item \textbf{Platform scalability:} Food-grade PDEV source (citrus juice waste, \$0.10--0.50/dose); established PLGA manufacturing; modular design extensible to other drugs/enzymes/diseases.
\end{enumerate}

\textbf{Significance:} Success will yield a lead formulation with comprehensive preclinical dataset supporting Phase II IND-enabling studies. Beyond ASUC rescue therapy (75,000 cases/year in US/EU), this platform addresses a \$2B+ global CsA market constrained by toxicity. If chronic safety is demonstrated, applications extend to IBD maintenance therapy (\$500M+ opportunity), graft-versus-host disease, and organ transplantation. Our approach establishes a new paradigm for combining natural (plant EV) and synthetic (PLGA) nanotechnologies with molecular-level activation triggers.

\textbf{Commercialization Strategy:} Phase I delivers formulation, mechanism-of-action data, and patent protection (provisional filed Month 1; full utility by Month 12). Phase II (Years 2--3) will complete GLP toxicology, process scale-up, and CMC for IND filing, targeting Phase IIa clinical trial in ASUC patients by Year 4. We have preliminary discussions with three potential pharma partners (Takeda, Ferring, AbbVie) interested in differentiated oral IBD therapies. FDA regulatory path: 505(b)(2) NDA leveraging existing CsA safety data.

% Milestones
\vspace{0.2cm}
\noindent\rule{\textwidth}{0.5pt}
\vspace{0.1cm}

\section*{GO/NO-GO MILESTONES}

\textbf{Month 4:} PDEV-PLGA-CsA formulations achieve $\geq$50\% NP encapsulation, structural integrity by TEM, and $\geq$80\% GI stability.

\textbf{Month 8:} PLA2-triggered NP release confirmed ($\geq$2-fold vs. control); PLA2-cleavable formulation shows $\geq$50\% potency improvement (lower IC$_{50}$) in T-cell assays.

\textbf{Month 12:} In vivo efficacy: $\geq$40\% DAI reduction with $<$50\% systemic CsA exposure vs. free CsA; no added toxicity; biodistribution confirms gut targeting and PLA2-dependent NP release.

\textbf{Decision:} Meeting all milestones $\rightarrow$ Phase II SBIR (GLP tox, scale-up, IND prep). Partial success $\rightarrow$ Optimize linker chemistry or PDEV source; resubmit. Failure $\rightarrow$ Pivot to dual-cargo (CsA + nucleic acid) or single-level PDEV platform.

\vspace{0.3cm}
\hrule
\vspace{0.2cm}

\noindent\textbf{Principal Investigator:} Maria Beatriz Herrera Sanchez, PhD -- Expert in extracellular vesicles, nanomedicine, and regenerative medicine with 15+ publications on EV isolation, characterization, and therapeutic applications (Google Scholar: h-index 10, 450+ citations). Founder and Director of ExoVitae Lab, specializing in plant-derived EVs for drug delivery and tissue engineering.\\[0.2cm]
\textbf{Key Personnel:} [Co-Investigator Name], PhD (PLGA formulation expert); [Consultant Name], MD (Gastroenterologist, IBD clinical translation).\\[0.2cm]
\textbf{Facilities:} Access to ultracentrifugation, nanoparticle tracking analysis (NTA), dynamic light scattering (DLS), transmission electron microscopy (TEM), LC-MS/MS, flow cytometry, and animal facilities for IBD models.

\end{document}
