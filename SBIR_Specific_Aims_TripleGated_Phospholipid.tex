\documentclass[11pt]{article}
\usepackage[utf8]{inputenc}
\usepackage[margin=1in]{geometry}
\usepackage{times}
\usepackage{setspace}
\usepackage{enumitem}
\usepackage{parskip}

\onehalfspacing

\begin{document}

\begin{center}
\textbf{\large Triple-Gated Plant Extracellular Vesicle–Phospholipid Nanoparticle Platform for Safer Cyclosporine A Therapy in Inflammatory Bowel Disease}
\end{center}

\vspace{0.5em}

\section*{Overview and Significance}

Cyclosporine A (CsA) is an effective rescue therapy for steroid-refractory ulcerative colitis, but its use is constrained by nephrotoxicity, neurotoxicity, a narrow therapeutic window, and pronounced pharmacokinetic variability. Existing oral CsA formulations lack gut-specific targeting and deliver high systemic exposure that drives toxicity while failing to maximize local mucosal immunosuppression. Prior nanotechnologies have improved solubility, sustained release, or targeting individually, but no existing system provides \textbf{multi-level control} using an all-phospholipid, polymer-free architecture.

\section*{Central Hypothesis}

An all-phospholipid, triple-gated delivery system—CsA-loaded phospholipid nanoparticles encapsulated within plant-derived extracellular vesicles (PDEVs) and tethered via PLA$_2$-cleavable phospholipid linkers—will enable:
(i) preferential accumulation in inflamed intestinal tissue via PDEV tropism,
(ii) enzyme-triggered nanoparticle release at disease sites where secreted PLA$_2$ is elevated, and
(iii) sustained local CsA delivery via phospholipid matrix-controlled kinetics,
resulting in superior therapeutic index (efficacy with $<$50\% systemic exposure) compared to conventional CsA formulations.

\section*{Platform Constraint}

All formulations in this project will be built \textbf{exclusively from true glycerophospholipids and plant EV lipids} as carrier components; no polymeric nanoparticle cores and no phospholipid–drug prodrugs will be used. CsA will be physically encapsulated, not covalently conjugated, to maintain a clear distinction from CsA-phospholipid prodrug prior art.

\vspace{0.5em}

\section*{Specific Aim 1: Develop and characterize CsA-loaded phospholipid nanoparticles and their incorporation into plant-derived extracellular vesicles}

\textbf{Rationale:} Establish a reproducible, fully phospholipid-based inner carrier and its encapsulation into PDEVs.

\textbf{Approach:}
We will formulate \textbf{CsA-loaded phospholipid nanoparticles} (liposomal/nanosphere formulations, $\sim$80–150 nm) using non-conjugated glycerophospholipids and 20–40 mol\% cholesterol. Two therapeutic compositions will be explored: (i) a \textbf{barrier-repair mix} (e.g., 30–40 mol\% PC with linoleic acid enrichment, 15–20 mol\% omega-3 PS, LPC/PC $<$0.15), and (ii) a \textbf{resolution-promoting mix} (e.g., 25–35 mol\% DHA-PC, 15–20 mol\% EPA-PE, reduced arachidonic acid content). CsA will be encapsulated by film hydration and downsizing. In parallel, PDEVs will be isolated from \textit{Citrus limon}, \textit{Zingiber officinale}, and \textit{Brassica oleracea} and characterized by NTA, DLS, TEM, and lipidomics. We will develop methods for loading CsA-phospholipid nanoparticles into PDEVs (co-incubation, extrusion, or mild sonication), then quantify nanoparticle encapsulation efficiency, structural integrity, and stability in simulated gastric and intestinal fluids. Lipidomic profiling (LC-MS/MS) will confirm enrichment of therapeutic phospholipids (target: 15–60 mol\% of total membrane phospholipids and $\geq$1.5-fold vs unmodified EVs).

\textbf{Milestones (Month 4):}
(i) CsA-phospholipid nanoparticles with $\geq$5–10 wt\% CsA loading and controlled release over 24–72 h in vitro;
(ii) PDEV–phospholipid nanoparticle complexes with $\geq$40–50\% nanoparticle encapsulation, preserved EV morphology, and stable lipidomic profiles over 4 weeks at 4 °C.

\vspace{0.5em}

\section*{Specific Aim 2: Engineer PLA$_2$-cleavable phospholipid linkers and demonstrate enzyme-triggered nanoparticle release and immunosuppressive activity}

\textbf{Rationale:} Validate the inflammation-responsive "Gate 2" and demonstrate that PLA$_2$ activity selectively liberates inner phospholipid nanoparticles at disease-relevant sites.

\textbf{Approach:}
We will design \textbf{PLA$_2$-sensitive phospholipid linkers} (sn-2–cleavable glycerophospholipids) that insert into PDEV membranes and tether internal phospholipid nanoparticles. Post-isolation linker insertion will be optimized, and incorporation quantified by LC-MS/MS. Enzyme-triggered release will be measured as nanoparticle liberation in the presence vs absence of recombinant sPLA$_2$-IIA and in homogenates from inflamed vs healthy mouse colon. We will then compare free CsA, CsA-phospholipid nanoparticles alone, non-cleavable PDEV–nanoparticle complexes, and PLA$_2$-cleavable PDEV–nanoparticle complexes in: (i) intestinal epithelial models (Caco-2, organoids) for uptake, TEER, FITC-dextran permeability, and tight junction protein expression; and (ii) T-cell–based assays (PBMCs) for NFAT activation and IL-2 secretion.

\textbf{Milestones (Month 8):}
(i) PLA$_2$ exposure induces $\geq$2–3-fold increase in nanoparticle release vs non-cleavable controls;
(ii) in inflamed epithelial models, PLA$_2$-cleavable PDEV–nanoparticle formulations restore TEER to $\geq$80\% of healthy control and reduce FITC-dextran flux by $\geq$50\% vs inflamed untreated controls;
(iii) in T-cell assays, PLA$_2$-cleavable formulations achieve equivalent IL-2 suppression at $\leq$50\% of the CsA concentration required for free CsA.

\vspace{0.5em}

\section*{Specific Aim 3: Validate multi-level targeting and therapeutic index in a murine colitis model}

\textbf{Rationale:} Demonstrate that the fully phospholipid triple-gated system improves therapeutic index in vivo.

\textbf{Approach:}
In DSS-induced colitis (C57BL/6 mice), we will compare oral administration of: (1) vehicle, (2) free CsA (10 mg/kg), (3) CsA-phospholipid nanoparticles, (4) non-cleavable PDEV–nanoparticle complexes, and (5) PLA$_2$-cleavable PDEV–nanoparticle complexes (target 5 mg/kg CsA-equivalent). Efficacy endpoints will include daily disease activity index, colon length, histopathology, mucosal cytokines, and barrier markers. PK and safety endpoints will include plasma CsA AUC, tissue CsA levels (colon vs kidney/liver), and nephrotoxicity/hepatotoxicity markers.

\textbf{Milestones (Month 12):}
(i) PLA$_2$-cleavable PDEV–nanoparticle group achieves $\geq$40\% reduction in disease activity index vs vehicle and is non-inferior to free CsA at 10 mg/kg;
(ii) plasma CsA AUC is $<$50\% of free CsA, with $\geq$3-fold higher colonic CsA concentration;
(iii) no significant increase in serum creatinine or kidney histopathology vs vehicle.

\vspace{0.5em}

\section*{Expected Impact}

Completion of Phase I will yield a lead, fully phospholipid triple-gated CsA formulation, mechanism-of-action data, and a strong IP position that deliberately avoids polymeric and CsA-phospholipid prodrug prior art, enabling a Phase II SBIR focused on GLP tox, scale-up, and IND preparation.

\end{document}
