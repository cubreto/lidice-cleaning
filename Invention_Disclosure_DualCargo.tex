\documentclass[11pt,letterpaper]{article}

% Packages
\usepackage[utf8]{inputenc}
\usepackage[margin=1in]{geometry}
\usepackage{times}
\usepackage{setspace}
\usepackage{graphicx}
\usepackage{hyperref}
\usepackage{enumitem}
\usepackage{bold-extra}
\usepackage[parfill]{parskip}
\usepackage{xcolor}

% Formatting
\setlength{\parindent}{0pt}
\setstretch{1.15}

\begin{document}

% Title Page
\begin{center}
{\Huge \textbf{INVENTION DISCLOSURE}}\\[0.5cm]
{\Large Dual-Cargo Plant Extracellular Vesicles for\\Programmable Oral Immunosuppression}\\[1cm]

\begin{tabular}{ll}
\textbf{Inventors:} & [Beatriz Name], PhD\\
& [Your Name], PhD\\[0.3cm]
\textbf{Institution:} & [Company / University Name]\\[0.3cm]
\textbf{Date:} & \today\\[0.3cm]
\textbf{Status:} & Confidential -- For Patent Counsel Review\\
\end{tabular}
\end{center}

\vspace{1cm}
\hrule
\vspace{0.5cm}

\tableofcontents
\newpage

% Section 1: Executive Summary
\section{Executive Summary}

\subsection{One-Line Description}
\textbf{A class of plant-derived extracellular vesicles that co-deliver cyclosporine A (or similar hydrophobic immunosuppressants) together with immunomodulatory nucleic acids (miRNA/siRNA/ASO/mRNA), enabling deeper, more targeted immune control at lower drug doses.}

\subsection{Key Innovation}
This invention transforms plant extracellular vesicles (EVs) from passive drug carriers into \textbf{programmable, bi-modal immunosuppressive platforms} by:
\begin{enumerate}[leftmargin=*]
    \item \textbf{Co-encapsulating two distinct therapeutic modalities:} A hydrophobic small molecule (CsA) in the lipid membrane + immunomodulatory nucleic acid in the EV lumen.
    \item \textbf{Achieving synergistic immunosuppression:} Targeting complementary pathways (calcineurin inhibition + transcriptional/post-transcriptional regulation) to enable dose reduction and improved safety.
    \item \textbf{Maintaining food-grade, scalable manufacturing:} Using waste streams from juice processing (citrus, grape, ginger) as EV source.
\end{enumerate}

\subsection{Differentiation from Prior Art}
\begin{itemize}[leftmargin=*]
    \item \textbf{Prior work on plant EVs:} Single-cargo delivery (drug OR nucleic acid, not both).
    \item \textbf{Prior work on CsA formulations:} Synthetic carriers (polymers, liposomes) without biological targeting or synergistic co-cargo.
    \item \textbf{Our invention:} First dual-cargo plant EV platform; first rational pairing of small-molecule immunosuppressant with pathway-specific nucleic acid.
\end{itemize}

\subsection{Target Applications}
\begin{itemize}[leftmargin=*]
    \item \textbf{Primary:} Acute severe ulcerative colitis (ASUC) -- 75,000 cases/year in US/EU, high unmet need.
    \item \textbf{Secondary:} Crohn's disease, graft-versus-host disease, organ transplant rejection, psoriasis.
    \item \textbf{Platform potential:} Extensible to other drug-nucleic acid pairs (tacrolimus, sirolimus, cannabinoids).
\end{itemize}

\newpage

% Section 2: Detailed Invention Description
\section{Detailed Invention Description}

\subsection{Background and Unmet Need}

\subsubsection{Cyclosporine A (CsA) in Inflammatory Bowel Disease}
Cyclosporine A is a potent calcineurin inhibitor widely used in transplantation and severe autoimmune/inflammatory diseases. In acute severe ulcerative colitis (ASUC), intravenous CsA achieves 60--80\% response rates as rescue therapy for patients failing corticosteroids. However, systemic toxicity limits its use:
\begin{itemize}[leftmargin=*]
    \item \textbf{Nephrotoxicity:} 20--40\% of patients develop renal impairment.
    \item \textbf{Neurotoxicity:} Tremor, paresthesias, seizures in 5--10\%.
    \item \textbf{Narrow therapeutic window:} Requires frequent drug monitoring.
    \item \textbf{Poor oral bioavailability:} Erratic absorption, food effects.
\end{itemize}

Current oral formulations (Neoral, Gengraf) are microemulsions that improve absorption but do not achieve gut-specific targeting, resulting in high systemic exposure.

\subsubsection{Nucleic Acid Therapeutics for Immune Modulation}
Antisense oligonucleotides (ASOs), siRNAs, and miRNA mimics/inhibitors can precisely modulate immune pathways (e.g., suppress IL-2, TNF-$\alpha$, IL-17; enhance FOXP3/Tregs). However, oral delivery is severely limited by:
\begin{itemize}[leftmargin=*]
    \item Rapid degradation by nucleases in GI tract.
    \item Poor epithelial permeability (hydrophilic, anionic).
    \item Off-target effects in non-diseased tissues.
\end{itemize}

No approved oral nucleic acid therapeutic exists for IBD.

\subsubsection{Plant Extracellular Vesicles (PDEVs) as Carriers}
Plant-derived EVs are naturally occurring nanoparticles (50--200 nm) secreted by plant cells. Citrus, grapefruit, ginger, and grape EVs have been shown to:
\begin{itemize}[leftmargin=*]
    \item Survive gastric pH and proteolytic enzymes.
    \item Cross intestinal epithelial barriers.
    \item Deliver bioactive cargo (proteins, lipids, nucleic acids) to mammalian cells.
    \item Exhibit low immunogenicity and GRAS (Generally Recognized As Safe) status potential.
\end{itemize}

However, \textbf{no prior work has systematically engineered PDEVs to co-deliver hydrophobic drugs and nucleic acids simultaneously}, nor designed synergistic drug-nucleic acid pairs for specific diseases.

\subsection{Inventive Concept}

\subsubsection{Core Innovation: Dual-Cargo Architecture}
We engineer plant EVs to simultaneously carry:
\begin{enumerate}[leftmargin=*]
    \item \textbf{Hydrophobic small molecule (CsA):} Inserted into the lipid bilayer membrane via temperature-controlled passive loading (``Gentle Fusion'' protocol).
    \item \textbf{Immunomodulatory nucleic acid:} Encapsulated in the EV lumen or associated with the inner membrane leaflet via electroporation, lipid-mediated transfection, or cationic polymer complexation.
\end{enumerate}

This dual-cargo system enables:
\begin{itemize}[leftmargin=*]
    \item \textbf{Orthogonal loading mechanisms:} Membrane-based loading for lipophilic drugs; active loading for nucleic acids.
    \item \textbf{Independent dose optimization:} CsA and nucleic acid ratios can be tuned for maximal synergy.
    \item \textbf{Compartmentalized release:} CsA releases via membrane diffusion in target tissue; nucleic acid releases upon EV internalization and endosomal escape.
\end{itemize}

\subsubsection{Rational Selection of Nucleic Acid Payloads}
For ASUC, we select nucleic acids targeting pathogenic immune pathways:

\begin{table}[h!]
\centering
\small
\begin{tabular}{|l|p{6cm}|p{5cm}|}
\hline
\textbf{Target} & \textbf{Nucleic Acid Modality} & \textbf{Rationale} \\
\hline
IL-2 mRNA & Antisense oligonucleotide (ASO) & Suppress T-cell activation synergistically with CsA (which blocks IL-2 transcription via NFAT inhibition). \\
\hline
TNF-$\alpha$ mRNA & siRNA or ASO & Reduce epithelial/macrophage inflammation; complement CsA's T-cell effect. \\
\hline
IL-17A/F & siRNA & Target Th17 pathway (not affected by calcineurin inhibitors). \\
\hline
FOXP3 & mRNA or miRNA mimic & Expand regulatory T cells (Tregs), enhancing immunosuppression while potentially reducing CsA dose. \\
\hline
\end{tabular}
\caption{Candidate nucleic acid payloads for dual-cargo PDEV-CsA formulations.}
\end{table}

\subsubsection{Synergistic Mechanism of Action}
\begin{itemize}[leftmargin=*]
    \item \textbf{CsA:} Inhibits calcineurin $\rightarrow$ blocks NFAT nuclear translocation $\rightarrow$ suppresses IL-2 transcription and T-cell activation.
    \item \textbf{Anti-IL-2 ASO:} Degrades IL-2 mRNA post-transcriptionally, providing a second layer of IL-2 blockade.
    \item \textbf{Net effect:} Deeper IL-2 suppression at lower CsA dose, reducing systemic exposure and toxicity.
\end{itemize}

Alternative combinations (e.g., CsA + anti-TNF siRNA) target \textbf{complementary pathways}, achieving broader immunosuppression.

\newpage

% Section 3: Detailed Manufacturing Method
\section{Detailed Manufacturing Method}

\subsection{Step 1: Plant EV Isolation}

\textbf{Source Material:} Fresh-pressed juice from edible plants (citrus, grapefruit, ginger, grape). Prioritize industrial waste streams for cost efficiency.

\textbf{Isolation Protocol:}
\begin{enumerate}[leftmargin=*]
    \item \textbf{Clarification:} Centrifuge juice at 500$\times$g (10 min) $\rightarrow$ 2,000$\times$g (20 min) $\rightarrow$ 10,000$\times$g (30 min) to remove cells, debris, and large vesicles.
    \item \textbf{EV purification:} Apply supernatant to size-exclusion chromatography (SEC) columns (e.g., qEV, Izon Science) or tangential flow filtration (TFF) with 100--300 kDa MWCO.
    \item \textbf{Concentration:} Collect EV-enriched fractions; concentrate via TFF or ultracentrifugation (100,000$\times$g, 90 min).
    \item \textbf{Quality control:} Measure particle concentration (NTA), size distribution (DLS), protein content (BCA), EV markers (Western blot: CD63, TSG101).
\end{enumerate}

\textbf{Yield:} $\sim$50--200 $\mu$g EV protein per mL juice; $\sim$10$^{10}$--10$^{11}$ particles per mL.

\subsection{Step 2: CsA Loading (``Gentle Fusion'' Protocol)}

\textbf{Rationale:} Maximize CsA encapsulation in the lipid membrane while preserving EV integrity and native cargo (flavonoids, proteins).

\textbf{Protocol:}
\begin{enumerate}[leftmargin=*]
    \item \textbf{Pre-activation:} Incubate purified EVs at 37°C for 15 min to increase membrane fluidity.
    \item \textbf{CsA introduction:}
    \begin{itemize}[leftmargin=*]
        \item Prepare CsA stock in ethanol (10 mg/mL).
        \item Add dropwise to EV suspension (final ethanol concentration: 2--5\% v/v; CsA:EV ratio: 1:2.5 to 1:10 w/w).
        \item Maintain 37°C for 30 min with gentle rotation (10 rpm).
    \end{itemize}
    \item \textbf{Membrane annealing:} Rapidly cool to 4°C (5 min ice bath) to ``lock'' CsA into membrane.
    \item \textbf{Purification:} Remove free CsA via SEC or ultracentrifugation (100,000$\times$g, 90 min, 4°C); wash pellet once.
    \item \textbf{Characterization:}
    \begin{itemize}[leftmargin=*]
        \item Encapsulation efficiency (EE\%): quantify CsA in purified EVs by LC-MS/MS; EE\% = (CsA in EVs / CsA added) $\times$ 100.
        \item Target: EE\% $\geq$ 60\%; loading capacity $\geq$ 50 $\mu$g CsA/mg EV protein.
    \end{itemize}
\end{enumerate}

\subsection{Step 3: Nucleic Acid Loading}

\textbf{Method A: Electroporation} (preferred for robustness)
\begin{enumerate}[leftmargin=*]
    \item Resuspend CsA-loaded EVs in low-conductivity buffer (e.g., 0.5$\times$ PBS or sucrose buffer).
    \item Add nucleic acid (ASO, siRNA, miRNA mimic) to EV suspension (molar ratio: 1:100 to 1:1000 nucleic acid:EV).
    \item Electroporate: 400 V, 125 $\mu$F, 2-mm cuvette (exponential decay pulse).
    \item Incubate 30 min at room temperature (allow membrane resealing).
    \item Purify via SEC or ultracentrifugation to remove free nucleic acid.
\end{enumerate}

\textbf{Method B: Lipid-Mediated Transfection}
\begin{enumerate}[leftmargin=*]
    \item Complex nucleic acid with cationic lipid (e.g., DOTAP, Lipofectamine) at 1:3 molar ratio.
    \item Incubate lipoplexes with CsA-loaded EVs for 1--2 h at 37°C.
    \item Purify as above.
\end{enumerate}

\textbf{Characterization:}
\begin{itemize}[leftmargin=*]
    \item Nucleic acid loading: quantify by qPCR (copies per $10^9$ EVs) or fluorescence (if labeled). Target: $>$10$^4$ copies per $10^9$ EVs.
    \item Integrity: Assess size, zeta potential post-loading (expect slight increase in size [10--20 nm], shift in zeta toward neutral/positive).
\end{itemize}

\subsection{Step 4: Formulation and Stabilization}

\textbf{Lyoprotectant Addition:} Add trehalose or sucrose (5--10\% w/v) to protect EVs during storage or lyophilization.

\textbf{Final Formulation:}
\begin{itemize}[leftmargin=*]
    \item \textbf{Liquid:} Store at 4°C (use within 4 weeks) or $-80$°C (long-term).
    \item \textbf{Lyophilized:} Freeze-dry for room-temperature storage; reconstitute with water before use.
\end{itemize}

\textbf{Quality Release Criteria:}
\begin{itemize}[leftmargin=*]
    \item CsA content: 90--110\% of label claim (HPLC/LC-MS).
    \item Nucleic acid content: $\geq$80\% of target loading (qPCR).
    \item Particle size: 80--250 nm (NTA).
    \item Sterility and endotoxin: USP <71>, <85> compliant (for clinical use).
\end{itemize}

\newpage

% Section 4: Patentable Claims
\section{Patentable Claims (Draft for Counsel)}

\subsection{Independent Claim 1: Composition of Matter}

\textbf{Claim 1.} A pharmaceutical composition comprising:
\begin{enumerate}[label=(\alph*),leftmargin=*]
    \item extracellular vesicles derived from an edible plant source;
    \item a hydrophobic calcineurin inhibitor associated with a membrane of the extracellular vesicles; and
    \item at least one immunomodulatory nucleic acid encapsulated within the extracellular vesicles or associated with an inner leaflet of the membrane,
\end{enumerate}
wherein the immunomodulatory nucleic acid modulates expression or activity of an immune-related target selected from the group consisting of interleukin-2 (IL-2), tumor necrosis factor-alpha (TNF-$\alpha$), interferon-gamma (IFN-$\gamma$), interleukin-17 (IL-17), forkhead box P3 (FOXP3), nuclear factor of activated T-cells (NFAT), and nuclear factor-kappa B (NF-$\kappa$B).

\subsection{Dependent Claims (Examples)}

\textbf{Claim 2.} The composition of Claim 1, wherein the edible plant source is selected from the group consisting of \textit{Citrus sinensis} (sweet orange), \textit{Citrus paradisi} (grapefruit), \textit{Citrus limon} (lemon), \textit{Citrus bergamia} (bergamot), \textit{Zingiber officinale} (ginger), and \textit{Vitis vinifera} (grape).

\textbf{Claim 3.} The composition of Claim 1, wherein the hydrophobic calcineurin inhibitor is cyclosporine A, tacrolimus, pimecrolimus, or a pharmaceutically acceptable salt thereof.

\textbf{Claim 4.} The composition of Claim 1, wherein the immunomodulatory nucleic acid is an antisense oligonucleotide (ASO), small interfering RNA (siRNA), microRNA mimic, microRNA inhibitor, or messenger RNA (mRNA).

\textbf{Claim 5.} The composition of Claim 1, wherein the immunomodulatory nucleic acid targets IL-2 mRNA for degradation.

\textbf{Claim 6.} The composition of Claim 1, wherein:
\begin{enumerate}[label=(\alph*),leftmargin=*]
    \item the mass ratio of cyclosporine A to immunomodulatory nucleic acid is between 10:1 and 1000:1; and
    \item the extracellular vesicles have a mean diameter between 50 nm and 300 nm.
\end{enumerate}

\textbf{Claim 7.} The composition of Claim 1, wherein the extracellular vesicles retain at least 40\% of endogenous bioactive compounds present in the plant source, selected from flavonoids, polyphenols, and terpenes.

\subsection{Independent Claim 2: Method of Treatment}

\textbf{Claim 8.} A method of treating an immune-mediated disease in a subject in need thereof, comprising orally administering to the subject an effective amount of the composition of Claim 1, wherein the immune-mediated disease is selected from the group consisting of ulcerative colitis, Crohn's disease, acute severe ulcerative colitis (ASUC), graft-versus-host disease, organ transplant rejection, psoriasis, rheumatoid arthritis, and multiple sclerosis.

\textbf{Claim 9.} The method of Claim 8, wherein the oral administration achieves:
\begin{enumerate}[label=(\alph*),leftmargin=*]
    \item a plasma area-under-the-curve (AUC) of the calcineurin inhibitor that is less than 70\% of the AUC achieved by an equivalent oral dose of free calcineurin inhibitor; and
    \item a tissue concentration of the calcineurin inhibitor in the colon that is at least 2-fold higher than the tissue concentration achieved by free calcineurin inhibitor.
\end{enumerate}

\subsection{Independent Claim 3: Method of Manufacture}

\textbf{Claim 10.} A method of producing a dual-cargo pharmaceutical composition, comprising:
\begin{enumerate}[label=(\alph*),leftmargin=*]
    \item isolating extracellular vesicles from an edible plant material by differential centrifugation and size-exclusion chromatography;
    \item contacting the extracellular vesicles with cyclosporine A in the presence of 2--5\% (v/v) ethanol at 37°C for 20--40 minutes, followed by rapid cooling to 4°C, to load cyclosporine A into a membrane of the extracellular vesicles;
    \item removing unencapsulated cyclosporine A by size-exclusion chromatography or ultracentrifugation;
    \item introducing an immunomodulatory nucleic acid into the extracellular vesicles by electroporation or lipid-mediated transfection; and
    \item removing free nucleic acid by purification,
\end{enumerate}
thereby obtaining extracellular vesicles co-encapsulating cyclosporine A and the immunomodulatory nucleic acid.

\textbf{Claim 11.} The method of Claim 10, wherein the encapsulation efficiency of cyclosporine A is at least 60\%, and the loading of immunomodulatory nucleic acid is at least 10$^4$ copies per 10$^9$ extracellular vesicles.

\newpage

% Section 5: Prophetic Examples
\section{Prophetic Examples}

\subsection{Example 1: Isolation and Characterization of Citrus EVs}

\textbf{Procedure:} Fresh sweet orange juice (2 L, organic) was subjected to sequential centrifugation (500$\times$g for 10 min, 2,000$\times$g for 20 min, 10,000$\times$g for 30 min) to remove pulp, cells, and debris. The clarified supernatant was applied to qEV size-exclusion columns (Izon Science, 70 nm separation). Fractions 7--9 were collected, pooled, and concentrated using 100 kDa MWCO spin filters.

\textbf{Expected Results:}
\begin{itemize}[leftmargin=*]
    \item Yield: 150 $\mu$g EV protein per mL juice; $\sim$5 $\times$ 10$^{10}$ particles/mL (NTA).
    \item Size: Mean diameter 120 $\pm$ 35 nm; polydispersity index (PDI) $<$ 0.25.
    \item Morphology: TEM imaging shows cup-shaped vesicles with intact bilayer membranes.
    \item Markers: Western blot positive for CD63, TSG101 (enriched vs. crude juice).
    \item Bioactive cargo: HPLC detects hesperidin (28 $\pm$ 6 $\mu$g/mg EV protein) and naringenin (12 $\pm$ 3 $\mu$g/mg).
\end{itemize}

\subsection{Example 2: CsA Loading via Gentle Fusion Protocol}

\textbf{Procedure:} Citrus EVs (1 mg protein) were pre-warmed to 37°C for 15 min. CsA (200 $\mu$g in 20 $\mu$L ethanol) was added dropwise with gentle swirling (final ethanol: 2\% v/v; CsA:EV ratio 1:5 w/w). The mixture was incubated at 37°C for 30 min, then rapidly cooled on ice for 5 min. Free CsA was removed by qEV column purification. CsA content in purified EVs was quantified by LC-MS/MS.

\textbf{Expected Results:}
\begin{itemize}[leftmargin=*]
    \item Encapsulation efficiency: 68 $\pm$ 5\% (n=3 batches).
    \item Loading capacity: 136 $\mu$g CsA per mg EV protein.
    \item Size: Increase from 120 nm to 135 nm post-loading; PDI remains $<$ 0.3.
    \item EV integrity: CD63 retention $>$ 90\% vs. empty EVs (Western blot).
    \item Stability: $>$ 90\% CsA retained at 4°C for 28 days.
\end{itemize}

\subsection{Example 3: Dual-Cargo Loading (CsA + Anti-IL-2 ASO)}

\textbf{Procedure:} CsA-loaded citrus EVs (from Example 2) were resuspended in 0.5$\times$ PBS. Anti-IL-2 antisense oligonucleotide (ASO, 20-mer 2'-O-methyl phosphorothioate) was added at 1:500 molar ratio (ASO:EV). Electroporation was performed (400 V, 125 $\mu$F, 2-mm cuvette). After 30 min recovery, free ASO was removed by SEC. ASO loading was quantified by qPCR.

\textbf{Expected Results:}
\begin{itemize}[leftmargin=*]
    \item ASO loading: 2.5 $\times$ 10$^4$ copies per 10$^9$ EVs.
    \item CsA retention: $>$ 85\% of pre-electroporation CsA content (minimal leakage).
    \item Size: 145 nm (slight increase due to ASO); zeta potential shifts from $-$20 mV to $-$10 mV.
    \item Dual-cargo stability: $>$ 80\% of both CsA and ASO retained at 4°C for 14 days.
\end{itemize}

\subsection{Example 4: In Vitro Synergistic T-Cell Suppression}

\textbf{Procedure:} Human PBMCs were activated with anti-CD3/CD28 beads and treated with: (1) vehicle, (2) free CsA (50 ng/mL), (3) CsA-EVs (50 ng/mL CsA-equivalent), (4) ASO-EVs (no CsA), (5) dual-cargo EVs (25 ng/mL CsA + ASO). IL-2 secretion was measured by ELISA at 48 h. Synergy was assessed using Bliss independence model.

\textbf{Expected Results:}
\begin{itemize}[leftmargin=*]
    \item Free CsA (50 ng/mL): 48 $\pm$ 6\% IL-2 suppression.
    \item CsA-EVs (50 ng/mL): 55 $\pm$ 7\% suppression (enhanced delivery).
    \item ASO-EVs: 22 $\pm$ 5\% suppression (modest effect alone).
    \item \textbf{Dual-cargo EVs (25 ng/mL CsA):} 72 $\pm$ 8\% suppression.
    \item Combination index: 0.65 (synergy; threshold for synergy $<$ 0.9).
    \item Interpretation: Dual-cargo achieves greater suppression at \textbf{half the CsA dose}.
\end{itemize}

\subsection{Example 5: In Vivo Efficacy in DSS Colitis Model (Conceptual)}

\textbf{Study Design:} C57BL/6 mice (n=10/group) receive 2.5\% DSS in drinking water for 7 days. On days 3--7, mice are orally gavaged with: (1) vehicle, (2) free CsA (10 mg/kg), (3) CsA-EVs (10 mg/kg CsA-equivalent), (4) dual-cargo EVs (5 mg/kg CsA + ASO).

\textbf{Expected Results:}
\begin{itemize}[leftmargin=*]
    \item Disease activity index (DAI, day 8):
    \begin{itemize}[leftmargin=*]
        \item Vehicle: 8.2 $\pm$ 1.1
        \item Free CsA: 4.5 $\pm$ 0.9 (45\% reduction)
        \item CsA-EVs: 4.0 $\pm$ 0.8 (51\% reduction; improved over free CsA)
        \item \textbf{Dual-cargo EVs: 3.2 $\pm$ 0.7 (61\% reduction; superior efficacy at half-dose)}
    \end{itemize}
    \item Plasma CsA AUC (0--6h):
    \begin{itemize}[leftmargin=*]
        \item Free CsA (10 mg/kg): 3,200 ng$\cdot$h/mL
        \item CsA-EVs (10 mg/kg): 2,100 ng$\cdot$h/mL (34\% lower)
        \item \textbf{Dual-cargo EVs (5 mg/kg): 1,100 ng$\cdot$h/mL (66\% lower)}
    \end{itemize}
    \item Colonic CsA concentration (terminal, $\mu$g/g tissue):
    \begin{itemize}[leftmargin=*]
        \item Free CsA: 12 $\pm$ 3
        \item CsA-EVs: 28 $\pm$ 6 (2.3-fold higher)
        \item Dual-cargo EVs: 22 $\pm$ 5 (1.8-fold higher; reduced dose but still gut-targeted)
    \end{itemize}
    \item Nephrotoxicity (serum creatinine, mg/dL):
    \begin{itemize}[leftmargin=*]
        \item Vehicle: 0.25 $\pm$ 0.05
        \item Free CsA: 0.42 $\pm$ 0.08 (68\% increase)
        \item CsA-EVs: 0.35 $\pm$ 0.06 (40\% increase)
        \item \textbf{Dual-cargo EVs: 0.28 $\pm$ 0.05 (12\% increase; not significant)}
    \end{itemize}
    \item Interpretation: Dual-cargo EVs achieve superior efficacy with minimal systemic toxicity by enabling dose reduction via synergistic mechanism.
\end{itemize}

\newpage

% Section 6: Commercial and Regulatory Strategy
\section{Commercial and Regulatory Strategy}

\subsection{Target Market}
\begin{itemize}[leftmargin=*]
    \item \textbf{Primary:} Acute severe ulcerative colitis (ASUC) rescue therapy -- 75,000 cases/year (US/EU), \$50--100M market.
    \item \textbf{Secondary:} Maintenance IBD therapy (if long-term safety demonstrated) -- \$500M+ market.
    \item \textbf{Tertiary:} Transplantation, GVHD, psoriasis -- \$2B+ total CsA market.
\end{itemize}

\subsection{Regulatory Path}
\begin{itemize}[leftmargin=*]
    \item \textbf{FDA 505(b)(2) NDA:} Leverage existing CsA safety data; demonstrate improved PK/safety profile.
    \item \textbf{GRAS status:} Seek affirmation for citrus EVs as food-grade excipient.
    \item \textbf{Timeline:} Phase I SBIR (Year 1) $\rightarrow$ Phase II (Years 2--3, IND-enabling) $\rightarrow$ Phase IIa clinical (Year 4) $\rightarrow$ NDA filing (Year 6).
\end{itemize}

\subsection{Licensing/Partnership Strategy}
\begin{itemize}[leftmargin=*]
    \item \textbf{Stage 1:} Option agreement with pharma partner after Phase I SBIR (\$500K--1M).
    \item \textbf{Stage 2:} Full license after IND filing (\$5--15M upfront + \$50--80M milestones + 5--10\% royalties).
    \item \textbf{Target partners:} Takeda, AbbVie, Pfizer, Ferring (companies with IBD portfolios).
\end{itemize}

\subsection{Platform Extension}
Beyond CsA, the dual-cargo platform can be adapted for:
\begin{itemize}[leftmargin=*]
    \item \textbf{Tacrolimus + anti-TNF siRNA} (transplant rejection, GVHD)
    \item \textbf{Cannabinoids + anti-inflammatory miRNAs} (IBD pain management)
    \item \textbf{Curcumin + FOXP3 mRNA} (cancer prevention, autoimmunity)
\end{itemize}

Each new drug-nucleic acid pair represents a separate licensing opportunity (\$5--10M per indication).

\newpage

% Section 7: Conclusion
\section{Conclusion}

This invention establishes a \textbf{new class of oral nanomedicines} that combine the advantages of plant-derived extracellular vesicles (biocompatibility, scalability, GRAS status) with programmable, synergistic co-delivery of small molecules and nucleic acids. By rationally pairing cyclosporine A with immunomodulatory nucleic acids targeting complementary pathways, we achieve:

\begin{itemize}[leftmargin=*]
    \item \textbf{Enhanced efficacy} via synergistic multi-pathway suppression.
    \item \textbf{Improved safety} through CsA dose reduction and gut-targeted delivery.
    \item \textbf{Platform scalability} enabling extension to multiple drug-disease pairs.
\end{itemize}

We recommend filing a provisional patent application immediately to establish priority, followed by SBIR Phase I grant application within 4--6 weeks. Preliminary data (plant EV characterization, CsA loading, dual-cargo feasibility) can be generated within 8--10 weeks to strengthen the full utility patent and Phase II SBIR proposal.

\vspace{1cm}
\hrule
\vspace{0.3cm}

\noindent\textbf{Inventors' Signatures:}\\[0.5cm]
\noindent\rule{6cm}{0.4pt} \hspace{1cm} Date: \rule{3cm}{0.4pt}\\
[Beatriz Name], PhD\\[0.5cm]

\noindent\rule{6cm}{0.4pt} \hspace{1cm} Date: \rule{3cm}{0.4pt}\\
[Your Name], PhD\\

\end{document}
